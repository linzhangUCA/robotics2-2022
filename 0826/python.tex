\documentclass[12pt,letterpaper]{beamer}
\usetheme{Copenhagen}
\usecolortheme{seahorse}
\setbeamertemplate{section in toc}{\inserttocsection}

\usepackage[utf8]{inputenc}
\usepackage{amsmath}
\usepackage{amsfonts}
\usepackage{amssymb}
\usepackage{graphicx}
\graphicspath{ {./images/} }
\usepackage{hyperref}
\hypersetup{
    colorlinks=true,
    linkcolor=blue,
    filecolor=magenta,      
    urlcolor=cyan,
    pdftitle={Overleaf Example},
    pdfpagemode=FullScreen,
}
\title[Robotics I]
{ENGR 3421: ROBOTICS I}
\subtitle{Into the Autonomous Ground Vehicle}

\author[Zhang, Lin]
{Dr. Lin Zhang}
\institute[UCA] % (optional)
{
  Department of Physics and Astronomy\\
  University of Central Arkansas
}
\date[Robotics1 2021] % (optional)
{August 24, 2021}
\logo{\includegraphics[height=1cm]{../images/uca_bear_logo.png}}


%End of title page configuration block
%------------------------------------------------------------

%------------------------------------------------------------
%The next block of commands puts the table of contents at the beginning of each section and highlights the current section:

% \AtBeginSection[]
% {
  % \begin{frame}
    % \frametitle{Outline}
    % \tableofcontents[currentsection]
  % \end{frame}
% }
%------------------------------------------------------------

\begin{document}

%The next statement creates the title page.
\frame{\titlepage}

%---------------------------------------------------------
%This block of code is for the table of contents after the title page
% \begin{frame}
% \frametitle{Outline}
% \tableofcontents
% \end{frame}
%---------------------------------------------------------


\begin{frame}{Introduction to Python}
    \begin{itemize}
        \item Using Python Interpreter
        \item Using Python as a Calculator
        \item Data Structures
        \item Control Flow
        \item Defining Functions
        \item Libraries
    \end{itemize}
\end{frame}

 
\end{document}
